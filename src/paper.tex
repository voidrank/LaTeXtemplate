% !TEX program = XeLaTeX
% !TEX encoding = UTF-8

\documentclass[UTF8]{ctexart}

% 页眉页脚
\usepackage{fancyhdr}
\pagestyle{fancy}
\lhead{} 
\chead{} 
\rhead{\bfseries \zihao{5} 14307130246 兰石懿 HIST119003.01 文艺复兴史} 
\lfoot{} 
\cfoot{}
\rfoot{\thepage} 
\renewcommand{\headrulewidth}{0.4pt} 
\renewcommand{\footrulewidth}{0.4pt}

% 脚注
\usepackage[marginal]{footmisc}
\renewcommand{\thefootnote}{\arabic {footnote} } %%脚注标号样式

%标题
\title{文艺复兴时期人文主义者价值观的理解}
\author{14307130246 兰石懿 HIST119003.01}
\date{\today}

\begin{document}
    \maketitle

    %%摘要
    \textbf{\zihao{+5}摘要}\quad
    \zihao{5}本文从欧洲文艺复兴时期人文主义者的新思想与中世纪传统基督教义与价值观的比较、人文主义者的理性追求、人文主义者对历史的影响三方面对文艺复兴时期人文主义矛盾复杂的思想进行粗浅的分析,旨在用现代人的价值观来理解文艺复兴时期人文主义者的思想,从而加深对欧洲文艺复兴的影响和深远意义的理解。

    %%关键词
    \textbf{\zihao{+5}关键字}\quad
    \zihao{5} 文艺复兴史\quad 人文主义者\quad 价值观\quad 经院哲学\quad 宗教改革\quad 西方启蒙运动\quad 中世纪基督教教义

    \textbf{\zihao{+5}前言}\par

    \zihao{5}文艺复兴时期的人文主义者在现代人看来是矛盾的。而这些现代人看起来的矛盾特性正是欧洲文艺复兴中起到了引导性的作用。
    因此,从现代价值观的角度来理解文艺复兴时期人文主义者的思想非常有助于我们更深层次地理解文艺复兴史形成发展的原因。\par

    人文主义者他们虽然没有抛弃对基督教,没有抛弃对上帝的信仰,甚至他们其中很多人本身就是神职人员。但他们其中也不乏批判教会,批判神职人员,批判教义的激进分子。
    人文主义者之间的矛盾也更是家常便饭。
    正如詹姆士·斯鲁威尔所说:“虽然许多历史学家,特别是19世纪的历史学家都试图把文艺复兴和意大利的人文主义与各种各样的非宗教形式联系起来,但并不是说,意大利的文艺复兴是非宗教的"\footnotemark[1]\par
    本文从欧洲文艺复兴时期人文主义者的新思想与中世纪传统基督教义与价值观的比较、人文主义者的理性追求、人文主义者对历史的影响三方面对文艺复兴时期人文主义矛盾复杂的思想进行粗浅的分析,旨在用现代人的价值观来理解文艺复兴时期人文主义者的思想,从而加深对欧洲文艺复兴的影响和深远意义的理解。



    \textbf{\zihao{+5}一、人文主义者的新思想与中世纪传统基督教义与价值观的比较}\par

    \zihao{5}
    虽说人文主义者的价值观也不都是完全相同,但是人文主义者比起传统基督教的价值观有着自己鲜明的特点。虽不能详尽地说明这些特点,但浅显地把两者作比较有助于我们加深对文艺复兴时期人文主义者思想的理解。\par
    1.中世纪基督教价值观特点:\par
    “没有人不渴望成功”这句话在今天看来非常正常。但在文艺复兴前的中世纪,这是不可想象的。
    基督教教义提出了关于人的本性的“原罪”之说(偷吃伊甸园中智慧树上的禁果而被上帝逐出伊甸园的人类的祖先亚当夏娃,违背了上帝的意志)。
    在基督教看来,所有人都是罪人。负罪成了每个人整个人生所要面对的共同义务和责任,并且这样的负罪是一代又一代继承下来的。
    因为负罪,人是没有权利去享乐的,人的一辈子都要用行动去弥补弥补这个罪恶。一切偷偷享乐的行为都是罪恶的。只有不断地向上帝忏悔,受到上帝的惩罚,最终才会得到救赎。
    基督教价值观中还带有强烈悲观处事的态度,这也与人文主义者的价值观相反。
    “基督教认为, 由于人类始祖犯下罪恶,人的本性就永远是罪恶的 , 受到上帝的惩罚, 所以人类在现世生活中无法逃避种种苦难的折磨, 无法救助自己脱离苦难。"\footnotemark[2]
    基督教通过对人类犯下原罪到最后获得救赎的整个过程的悲观主义描述, 彻底否定了人类的现世生活的自由和幸福, 彻底否定了人类掌握自己命运的任何可能性。\par
    2.人文主义者新价值观特点及与传统价值观的比较\par
    人文主义者在这个时代则抛弃了传统宗教观念,他们渴望成功,渴望声名鹊起,赞美世俗生活。他们肯定世俗的生活创造者和主人,他们要求文学艺术表现人的思想与感情,要求把思想、感情、智慧都从神学的束缚中解放出来。\par
    他们赞美人的伟大,认为人生来就伟大并且人人平等,提倡人的价值和尊严,强调人能够通过自身的努力来改变命运,认为改变世界的是人不是上帝。他们歌颂人的高贵品质,歌颂人的智慧,歌颂凡人的幸福生活。
    英国举世闻名的人文主义者莎士比亚在他的哈姆雷特剧作中, 尽情地赞美人的高贵与尊严: “人是多么了不起的一件作品! 理想是
多么高贵, 力量是多么无穷! 仪表和举止是多么端正, 多么
出色, 论行为, 多么像天使, 论了解, 多么像天神! 宇宙的精
华, 万物的灵长!"\footnotemark[3]
    他们还认为,人就应该享受现世的幸福生活,追求爱情雨物质是人的天性。呼吁人们热爱生活。
    艺术家们纷纷通过艺术作品来表现人的伟大,他们运用裸体来表现对人的赞美。他们崇尚古希腊罗马的审美标准,认为人体是力与美的结合,裸体画表现了肌肉之间的自然衔接之美,人体与人体之间相似而又完全不同的奥秘。这些裸体不仅展现丰富的美学,还表现了对人的意志和意识赞美。“人是肉体与灵魂、物质与精神的统一体,是审美主体和审美客体的集合物,是真善美的化身。”\footnotemark[4]
    其中一些人文主义者甚至站出来公然反对禁欲主义,揭露教会的腐败与虚伪。也正是这些矛盾冲突让更多人产生对教义这种“真理”的思考,使得更多的人加入到人文主义者的行列中来,从而推进了新的思想的萌芽。而不论是公然的反抗还是委婉地用艺术形式改革,这种思想已完全不是传统基督教那种毫无理性毫无人性的模样。\par
    

    \textbf{\zihao{+5}二、人文主义者的理性追求}\par
    \zihao{5}
    宗教改革的发生有很多的历史背景及原因,但这也和人文主义者思想新的思维观念有关系。\par
    
    1.中世纪基督教哲学\par
    中世纪基督教哲学被统称为经院哲学。经院哲学早期是对圣经、信条以及注释的分析得以发展。
    “圣托马斯·阿奎纳是经院哲学的代表人物,是自然神学的主要奠基人。
    他创造性的将同时代正在兴起的自然科学和当时占统治地位的天主教神学通过亚里士多德的逻辑分析方法结合起来,对神学和自然科学的各自发展产生重要影响。
    他认为自然科学为可通过学习、教育获得的低级感知,而作为高级感知的神学只能由上帝偶尔闪现的神迹和启示来获得,这就为神学和自然科学并存提供了可能。”\footnotemark[5]
    之后经院哲学家们通过理性形式的,抽象的,繁琐的辩证方法完善了这种哲学体系,并以各种各样的形式,“充分地”论证了基督教教义、信条的普世性。\par
    “虽然在经院哲学中已经能看到理性的影子,但由于经院哲学的最终目的是巩固基督教的统治地位,论证基督教信仰,为宗教服务,他有着自己的局限性。"\footnotemark[6]
    在很多理性和教义相矛盾的地方,经院哲学家选择了向教义妥协的态度。
    但反过来想,人文主义者的理性辩证思维和经院哲学的思想是分不开的。甚至可以说,人文主义者的新思维正是从经院哲学中吸取出来并对经院哲学的改革产生的。\par
    2.人文主义者的理性追求\par
    当时的人文主义者,虽然对理性的追求比之前的经院哲学家更甚。
    但不代表人文主义者是一个完全的无神论者。他们并没有在从事人文主义学术与创作的事后把自己放在上帝的对立面。
    他们想通过理性去批判反抗教会的腐败肮脏的意识和文化形态。“他们要求理性高于信仰,建立朴素而又符合理性的宗教,把宗教发挥到自然神的层次,成为一种维护社会秩序的化身”\footnotemark[7]
    他们的理性思维一方面体现在他们的批判思维和怀疑主义上。人文主义者批判经院哲学家僵化的学风,怀疑经院哲学家对圣经和教义的理解。其中不乏因与教会争论教义而死的人文主义者。这种批判和怀疑在经院哲学家中也曾有过,但人文主义者的怀疑的形式和内容更为激进而丰富。\par
    另一方面他们的理性体现在他们认为学术的平等性,他们建立了自主平等的学士探究原则,反对教会在圣经解释与理解上的垄断权。
    他们认为人人都可以用自己的方式去理解圣经。
    也只有通过更多人地阅读与平等地交流学习,而不是单方面让教会地灌输才能使基督教发扬光大。\par
    他们甚至想用圣经的神圣权威来挑战教皇的权威。这种形式的对哲学地应用已不再是经院哲学家那样是为了巩固基督教的权威性而存在了。人文主义者更多的是想用理性,从不可动摇的圣经出发,去改变甚至去击垮腐朽肮脏的教会,从而达到自己对人性追求的目的。
    甚至可以说,人文主义者借用了经院哲学家的理性来攻击经院哲学的建筑。但这种"攻击"实际上正是在推动整个文艺复兴发展,是极具意义的。\par

    \textbf{三、人文主义者对历史的影响}\par
    1.人文主义者对宗教改革的影响\par
    “宗教改革(英语:Protestant Reformation)是西方基督教在16世纪至17世纪的教派分裂,由約翰·威克里夫、扬·胡斯、馬丁·路德、約翰·加爾文以及其他早期新教徒发起。1517年,路德发表的《九十五条论纲》引发了宗教改革的开始。改革者反对当时罗马天主教的教条,仪式,领导和教会组织结构。在他们的努力下,新的国家性的改革派教會被建立。早期的一些发生在欧洲的事件(如黑死病的蔓延和天主教會大分裂)侵蚀了人们对天主教会和教皇的信仰,但教義上的歧見才是引发宗教改革的關鍵。其他一些因素(如文艺复兴思想的传播,印刷术的传播,东罗马帝国的灭亡)也都促成了新教的创立。"\footnotemark[8]\par
    人文主义者对理性的追求推动了更多人来参与教义的讨论,教义不再是之前教皇教会的统治工具,而是成为了很多人参与争论讨论的终极奥义。也正是这样的讨论揭露了教会的愚昧的见解,也正是这样的分歧推动了宗教改革。这种理性思维实际上间接地引发了宗教改革,甚至可以说是宗教改革的核心动力。
    另一方面,人文主义者对人本主义的宣扬让更多的人开始思考人神的关系。
    人们开始慢慢理解到,人本身就是美的,人生下来就有追求幸福的权利。没有人有必要为了教会宣扬的“原罪”去买单。这样的思维是让当时的人们挣脱了禁欲主义的束缚。
    这些束缚的减弱衰亡间接地导致了人们对教会的怀疑。另一方面,教会神职人员也被这种“世俗”的观念所影响。布克哈特在其著作《意大利文艺复兴时期的文化》中描述,“他们欺骗,偷到和四通,而当他们用尽了一切手段之后,他们装成圣徒假造奇迹……其他的人带着他们的同党,假装瞎子或假装受着致命的疾病的折磨的人,当着群众,触摸僧人们的袈裟的边缘或者他们所带的圣物,然后就恢复正常的样子,假装被治好了疾病。”\footnotemark[9]\par
    神职人员和普通大众的世俗化一下打破了之前基督教统治下的平静。教会方面极力想通过改革来挽回这种不正的风气,而其他包括人文主义者在内的人则越来越仇视神职人员,希望改革。\par
    2.人文主义者对西方启蒙运动的影响\par
    可以说欧洲文艺复兴时期运动无疑应该被看做萎新时代的开端和启蒙运动值滥觞。作为启蒙运动的精神先去,人文主义者身上有一种很强烈的怀古风,他们借用古希腊罗马文化,发展出新时代特有的自由主义气质和个人主义精神。
    他们借用这些精神加之对宗教教义和信条的推理与教会抗争。
    启蒙运动家便是在这样的文化基础和精神思想上发展,再借批判人文主义者保守的观念和传统的思想发展出启蒙运动。\par
    但也由于人文主义者的思想变革的不彻底性,阻碍了科学的发展。罗素认为,文艺复兴时代尽管在文学艺术上的成就琳琅满目, 但是在科学和哲学方面却是一个 “不毛的” 世纪,到了 17 世纪,尊重科学才成为“大多数重要革新 人物的特色”\footnotemark[10]
    


    \footnotetext[1]{\zihao{-5}詹姆士·斯鲁威尔[美].西方无神论简史}
    \footnotetext[2]{\zihao{-5}高飞乐.中世纪欧洲神学价值观的基本性质与特征.长白学刊CBXK.1999.1}
    \footnotetext[3]{\zihao{-5}威廉·莎士比亚[英].哈姆雷特}
    \footnotetext[4]{\zihao{-5}方放.琐谈裸体艺术的美学价值}
    \footnotetext[5]{\zihao{-5}维基百科.经院哲学}
    \footnotetext[6]{\zihao{-5}百度百科.经院哲学}
    \footnotetext[7]{\zihao{-5}孙继静.文艺复兴的信仰与理性.内蒙古农业大学学报(社会科学版).2004年第二期}
    \footnotetext[8]{\zihao{-5}维基百科.宗教改革}
    \footnotetext[9]{\zihao{-5}雅各布·布克哈特.意大利文艺复兴时期的文化[M].长川某译.成都:四川人民出版社.1989.253}
    \footnotetext[10]{\zihao{-5}罗素.西方哲学史.马原德译.北京:商务印书馆.1976年}
\end{document}

